% vim:nojs:spelllang=en_us tw=76 sw=4 sts=4 fo+=awn fmr={-{,}-} et ts=8
% COMP3241/9245 Esterel Component---Lab 2
% (Adapted for Zélus)
%
% 20051025 T.Bourke
%     Created original document
%
% 20120412 T.Bourke
%     Updated for Zélus.
\documentclass[12pt]{article}

%{-{1 Preamble

\usepackage[T1]{fontenc}
\usepackage[utf8]{inputenc}
\usepackage{lmodern}
\usepackage[
    pdfborder={0 0 0},
    pdfauthor={Timothy Bourke},
    pdftitle ={Backhoe Lab Exercise}
    pdfsubject={Real-time programming}
    pdfkeywords={synchronous languages, real-time, programming}
]{hyperref}
\usepackage{alltt}
\usepackage{graphicx}
\usepackage[top=25mm,bottom=25mm]{geometry}

\usepackage{tikz}
\usetikzlibrary{calc,positioning}

\setlength{\parindent}{0em}
\setlength{\parskip}{1ex plus 1pt minus 1pt}

\newcounter{task}
\newcommand{\nexttask}[1]{%
	\addtocounter{task}{1}%
	\textbf{\large{}Task \thetask: #1\ }\rule{0em}{4ex}}

\newenvironment{commands}%
	{\hfill\begin{minipage}{.9\textwidth}\begin{alltt}}
	{\end{alltt}\end{minipage}\hfill\newline}

% 1 - width (e.g., .95\textwidth); 2 - image name/path
\newenvironment{hgraphicscope}[2]
  {
    \node[anchor=south west,inner sep=0] (image) at (0,0)
       {\includegraphics[width=#1]{#2}};
    \begin{scope}
    \clip (image.south west) rectangle (image.north east);
    \path let \p1=(image.south east) in (\y1, \x1) coordinate (ylimit);
    \end{scope}
    \begin{scope}[x={(image.south east)},y={(ylimit)},scale=0.1,]
  }
  {
    \end{scope}
  }

%--   }-}1%%%%%%%%%%%%%%%%%%%%%%%%%%%%%%%%%%%%%%%%%%%%%%%%%%%%%%%%%%%%
\begin{document}

\begin{center}\bfseries
\LARGE{Backhoe loader exercises}
\end{center}

This lab consists of a series of programming exercises designed to increase 
familiarity and understanding of synchronous dataflow programming.
The example considered is a simplified backhoe loader.

The sections marked optional are more difficult.
They are recommended for students who find the other exercises easy.

\nexttask{Understanding the interface specification} %{-{1

\begin{figure}[ht]
\centering
\begin{tikzpicture}%{-{2
\begin{hgraphicscope}{.95\textwidth}{backhoe_loader}
    \tikzset{every path/.style={line width=2,dashed,color=black,line cap=round},
             every node/.style={font=\ttfamily\footnotesize, inner sep=1}}
    %\draw[help lines,xstep=1,ystep=1] (0,0) grid (image.north east);

    \draw
        (5.4,.7) -- +(0,1) node[above] {boom\_in}
        +(0,0) -- +(.5,-.5) node[right] {boom\_out}
        ;
    \draw
        (7.6,2.2) -- +(.6,.3) node[above] {stick\_in}
        +(0,0) -- +(-.15,-.7) node[below] {stick\_out}
        ;
    \draw
        (9.27,1.67) -- +(.9,-.05) node[above left=.02 and -.02] {bucket\_in}
        +(0,0) -- +(-.55,-.65) node[below] {bucket\_out}
        ;
    \draw
        (4.7,0.27) node (legsin) {} -- +(0.4,0)
        ++(0,-.27) node (legsout) {} -- +(0.4,0)
        ;
    \draw[thick,solid]
        (legsin) -- ++(-.45,-.45) -- +(-.2,0) node [left] {legs\_in}
        (legsout) -- ++(-.45,-.45) -- +(-.2,0) node [left] {legs\_out}
        ;
    \draw[thick,solid]
        (4.05,1.65) -- +(-1.5,.6)
            node[left,align=center]
            {stop\_button \\ extend\_button \\ retract\_button}
        ;
    \path[thick,solid,draw=black,fill=lightgray]
        (9.8,-0.15) circle (.3)
        node [left=.035] {second}
        +(0,0) -- +(110:.24)
        +(0,0) -- +(-30:.20)
        ;
\end{hgraphicscope}
\end{tikzpicture}%}-}2$
\caption{Backhoe Loader input signals\label{fig:backhoe}}
\end{figure}

Backhoe loaders are essentially tractors fitted with a loader unit at front, 
and a backhoe at rear~\cite{BrainHar:BackhoeLoaders}.
The backhoe consists of three segments: the \emph{boom}, \emph{stick} and 
\emph{bucket}.
Drivers are able to swivel their seats and coordinate the segments using 
joystick controls.

In this lab we will program a prototype sequencing controller to automate 
some aspects of backhoe operation.
The controller will receive input from the 12 signals shown in 
Figure~\ref{fig:backhoe}.
Each segment is fitted with two sensors, one triggered when the piston 
driving the segment is at its minimum (\texttt{$\ast$\_in}), the other when 
at its maximum (\texttt{$\ast$\_out}).
Similar sensors exist for a pair of stabilizing legs.
The cabin is fitted with three buttons, \emph{stop}, \emph{extend}, and 
\emph{retract}.
A signal is associated with each.
The controller also receives a timer input, \texttt{second}, at regular 
intervals.

\begin{figure}\centering
    \begin{tabular}{llll}
            \texttt{boom\_push} & \texttt{boom\_pull}
                                    & \texttt{boom\_drive} \\
            \texttt{stick\_push} & \texttt{stick\_pull}
                                    & \texttt{stick\_drive} \\
            \texttt{bucket\_push} & \texttt{bucket\_pull}
                                    & \texttt{bucket\_drive} \\
            \texttt{legs\_extend} & \texttt{legs\_retract}
                                    & \texttt{legs\_stop} \\
            \texttt{alarm\_lamp(bool)} & \texttt{done\_lamp(bool)}
                                    & \texttt{cancel\_lamp(bool)}
    \end{tabular}
    \caption{Backhoe Loader output signals\label{fig:outputs}}
\end{figure}

The controller is fitted with the 15 output signals summarized in 
Figure~\ref{fig:outputs}.
The segments are fitted with hydraulic pistons that either push or pull 
depending on the position of an internal valve.
The controller can position the valve by emitting either a 
\texttt{$\ast$\_push} or \texttt{$\ast$\_pull} signal (the effect of 
simultaneous emission is undefined).
Segments move clockwise when pushed by the hydraulics, and counter-clockwise 
when pulled.
Each segment has a closed-loop controller that, by default, holds it in 
fixed position, or, when a \texttt{$\ast$\_drive} signal is true, moves it 
at constant speed.
Motion ceases as soon as a signal becomes absent.
The legs operate differently.
The \texttt{legs\_extend} and \texttt{legs\_retract} signals start the legs 
moving until they are out or in, respectively.
This movement may be canceled at any time with the \texttt{legs\_stop} 
signal.
Three lamps, \emph{alarm}, \emph{done}, and \emph{cancel}, in the cabin are 
switched on and off by logical (true/false) signals.

%--   }-}1%%%%%%%%%%%%%%%%%%%%%%%%%%%%%%%%%%%%%%%%%%%%%%%%%%%%%%%%%%%%
\nexttask{Running the backhoe simulation} %{-{1

The simulation comprises a number of files:

\begin{tabular}{ll}
    \texttt{backhoegui.*} &
        displays the system state \\

    \texttt{backhoedyn.zls} &
        describes the backhoe dynamics (the plant) \\

    \texttt{backhoecontrol.zls} &
        contains the discrete controller\\
\end{tabular}

\noindent
The controller is implemented in \texttt{backhoecontrol.zls}.
It is compiled and linked with the other modules by typing \verb+make+.

When the resulting binary is executed, it displays the control and 
simulation windows.
The simulation window displays the backhoe and, at top-left, the Backhoe 
Loader and cabin interface (buttons and indicator lamps).
The control window contains buttons for pausing and stepping the simulation.
A simulation can be paused initially by executing the binary with the 
\verb+-pause+ option.

%--   }-}1%%%%%%%%%%%%%%%%%%%%%%%%%%%%%%%%%%%%%%%%%%%%%%%%%%%%%%%%%%%%
\nexttask{Extending the stick, then the bucket} %{-{1

Implement a simple controller (in \texttt{backhoecontrol.zls}) that first 
extends the \emph{stick} segment, and then the \emph{bucket} segment.
The segments must not be in motion simultaneously.
The \emph{done} lamp should be illuminated when the movements are complete.

%--   }-}1%%%%%%%%%%%%%%%%%%%%%%%%%%%%%%%%%%%%%%%%%%%%%%%%%%%%%%%%%%%%
\nexttask{Retracting the stick and bucket together} %{-{1

Modify the previous program so that (only) after the \emph{stick} and 
\emph{bucket} are fully extended, pushing the \emph{retract} button causes 
both to be retracted simultaneously.
The \emph{done} lamp should \emph{not} be illuminated while the backhoe is 
moving, but should come on again as soon as the motion is complete.

%--   }-}1%%%%%%%%%%%%%%%%%%%%%%%%%%%%%%%%%%%%%%%%%%%%%%%%%%%%%%%%%%%%
\nexttask{Flashing light during movement} %{-{1

Modify the previous program so that the \emph{alarm} light blinks on and off 
with a period of two seconds whenever the backhoe is moving.

%--   }-}1%%%%%%%%%%%%%%%%%%%%%%%%%%%%%%%%%%%%%%%%%%%%%%%%%%%%%%%%%%%%
\nexttask{Leg Control} %{-{1

Create a copy of the previous file, and start a new program.
Allow the three buttons to control leg movement directly, i.e., pressing 
\emph{extend} should start lowering the legs, \emph{stop} should stop them, 
and \emph{retract} should cause them to lift.
The controller should illuminate the \emph{done} lamp and halt if/when the 
legs are fully extended.

\textbf{Optionally:}
You can assume that the legs are always raised when the controller starts.
It should now be possible for the operator to raise, stop and lower the legs 
any number of times.
To avoid strain on the equipment, it is desirable to ensure that the legs 
are fully lowered within 10 seconds of the extend button having been pressed 
for the first time.

Add a watchdog timer in parallel to the leg control loop.
If, after 10 seconds have elapsed, the legs are not fully extended the alarm 
lamp should be illuminated and the legs should be completely raised, they 
should then be held for a further 5 seconds before the alarm lamp is 
switched off and normal operation resumes---all operator
input is to be ignored during the alarm period.
The watchdog timer can be reset each time the legs are fully extended or 
retracted.

%--   }-}1%%%%%%%%%%%%%%%%%%%%%%%%%%%%%%%%%%%%%%%%%%%%%%%%%%%%%%%%%%%%
\nexttask{Dig your own hole} %{-{1

Enhance the previous program.
Pressing \emph{extend} once the legs have been lowered should result in a 
series of movements:
\begin{enumerate}
\item Raise the \emph{stick} and \emph{bucket}.
\item Then, completely lower the \emph{boom} (into the ground).
\item Then, retract both the \emph{stick} and \emph{bucket}.
\item Wait for operator to press \emph{retract}.
\item Bring the \emph{boom} back into the starting position.
\end{enumerate}

During the first two steps pushing the \emph{retract} button should 
illuminate the \emph{cancel} lamp and retract all three segments 
(simultaneously) to the starting position.
The other buttons may be ignored during the retraction sequence.
The \emph{retract} button is to be ignored in steps 3 and 5.

Pressing the \emph{stop} button at any time should halt all movement.
The interrupted motion should resume when the button is pressed a second 
time.
The \emph{alarm} lamp should be illuminated during such interruptions, and 
the other buttons may be ignored.

The \emph{done} lamp should only be illuminated when the backhoe is 
stationary and the controller is waiting for operator input.
It should not, however, be illuminated when in the leg movement mode.

After the backhoe returns to the starting position, the buttons should 
control the legs again, as per the previous task.
At this stage all of the lamps should be dimmed.
The digging motion may be resumed when the legs are fully lowered, i.e., 
pressing \emph{extend} once would illuminate the \emph{done} lamp and 
pressing it again would start the movement sequence.

\textbf{Optionally:} Flash the \emph{cancel} lamp as the three segments are 
being retracted.

\textbf{Optionally:} The controller is damaging the valve seals by applying 
the throttle too soon after the direction has been changed.
Have the controller pause for at least three seconds between changing the 
valve direction and applying the drive signal so as to increase the 
longevity of the prototype equipment.
\textbf{warning:} Minimizing the delay is quite challenging (i.e. time 
consuming) if one considers interactions with the stopping feature, and 
`expectant' direction changes.

\textbf{Reflection:} Consider the issues/work involved if implementing the 
same behavior using an explicit (flat) Finite State Machine or a C program.

%--   }-}1%%%%%%%%%%%%%%%%%%%%%%%%%%%%%%%%%%%%%%%%%%%%%%%%%%%%%%%%%%%%

\bibliography{backhoe}
\bibliographystyle{abbrv}

\end{document}
