\documentclass[a4paper]{article}

\usepackage[T1]{fontenc}
\usepackage[prompt=,prefix=example]{runverbatim}
\usepackage{listings}

\lstdefinestyle{camlstyle}{%
  language=caml,
  basicstyle=\sffamily,
  keywordstyle=\bf,
  literate={"}{\textquotedbl}1,
}
%BEGIN LATEX
\runverbatimsetup{codelst={style=camlstyle}}
%END LATEX
%HEVEA \newstyle{.runverbatimerr}{font-style: italic; color: red;}
%HEVEA \newstyle{.runverbatimmsg}{font-style: italic;}
%HEVEA \runverbatimsetup{codelst=\{style=camlstyle\}}

\begin{document}

\noindent
Two simple declarations:
\begin{runverbatim}
let x = 3
let f x = x + 1
\end{runverbatim}

\noindent
and, after compilation with \runverbatimcmd, their types:
\runverbatimmsg{}

\noindent
Then, building on these definitions:
\begin{runverbatim}[continue,withresult,label=early]
let y = f 3
\end{runverbatim}

\noindent
What about compiler error messages?:
\begin{runverbatim}[fail]
let z = 3 + "four"
\end{runverbatim}

\noindent
They are also detected:
\runverbatimerr{}

\noindent
We can continue on from earlier examples:

\begin{runverbatim}[withresult,include=early]
let z = f y
\end{runverbatim}

\end{document}
